\documentclass[a4paper, 12pt]{article}
\usepackage[T2A]{fontenc}
\usepackage[utf8]{inputenc}
\usepackage[russian]{babel}
\usepackage{graphicx}
\usepackage{hyperref}
\usepackage{float}
\usepackage{amsmath}
\usepackage{geometry}
\usepackage{booktabs}
\usepackage{enumitem}
\geometry{margin=1in}

\title{Отчет по проверке гипотез с использованием случайных графов}
\author{Равиль Гареев}
\date{\today}

\begin{document}

\maketitle

\section*{Часть I: Проверка гипотез с использованием случайных графов}

\section*{Введение}
В работе исследуется применение случайных графов (KNN-графов и дистанционных графов) для проверки гипотез согласия. 
Цель — определить, насколько характеристики графов позволяют различать выборки из двух распределений: \(\chi^2\) (гипотеза \(H_0\)) и \(\chi\) (гипотеза \(H_1\)).

\section{Описание кода}

\subsection{Используемые инструменты}
\begin{itemize}
    \item \textbf{Python 3.10+}: Базовый язык разработки с строгой типизацией
    \item \textbf{Библиотеки}:
    \begin{itemize}[leftmargin=*]
        \item \texttt{numpy}: Векторизованные вычисления и работа с массивами
        \item \texttt{scipy.stats}: Генерация $\chi^2$ и $\chi$ распределений
        \item \texttt{scikit-learn}: Оптимизированное построение KNN-графов
        \item \texttt{networkx 3.0+}: Топологический анализ и алгоритмы на графах
        \item \texttt{matplotlib/seaborn}: Визуализация распределений характеристик
        \item \texttt{tqdm}: Интерактивные прогресс-бары для длительных вычислений
    \end{itemize}
    \item \textbf{Архитектура}: Модульная структура с разделением на генерацию данных, построение графов и анализ
\end{itemize}

\subsection{UML-диаграмма класса GraphAnalyzer}
\begin{figure}[H]
    \centering
    \includegraphics[width=0.75\textwidth]{uml_diagram.png}
    \caption{Диаграмма класса GraphAnalyzer с методами анализа}
    \label{fig:uml_graph}
\end{figure}

\subsection{Реализованные компоненты}
\begin{itemize}
    \item \textbf{Генераторы данных (distribution\_generators.py)}:
    \begin{itemize}
        \item $\chi^2$-распределение: Адаптер для \texttt{chi2.rvs()} с параметрами:
        \begin{itemize}
            \item \texttt{nu} - степени свободы
            \item \texttt{n} - размер выборки
        \end{itemize}
        \item $\chi$-распределение: Обертка для \texttt{chi.rvs()} с аналогичными параметрами
    \end{itemize}
    
    \item \textbf{Построители графов (build\_graph.py)}:
    \begin{itemize}
        \item \textbf{KNN-граф}:
        \begin{enumerate}
            \item Поиск $k+1$ ближайших соседей через \texttt{NearestNeighbors}
            \item Фильтрация петель (i $\neq$ j)
            \item Сохранение координат в атрибуте узлов
        \end{enumerate}
        
        \item \textbf{Дистанционный граф}:
        \begin{enumerate}
            \item Полный перебор всех пар вершин
            \item Проверка условия $|x_i - x_j| \leq d$
        \end{enumerate}
    \end{itemize}

    \item \textbf{Анализатор графов (graph\_analyzer.py)}:
    \begin{itemize}
        \item Расчёт степеней вершин: \texttt{max\_degree()}, \texttt{min\_degree()}
        \item Компоненты связности: \texttt{connected\_components()}
        \item Топологический анализ: \texttt{articulation\_points()}, \texttt{count\_triangles()}
        \item Раскраска графов: адаптивный алгоритм DSATUR в \texttt{chromatic\_number()}
        \item Клики: Алгоритм двух указателей для 1D в \texttt{clique\_number()}
        \item Оптимизационные задачи: независимые множества (\texttt{max\_independent\_set()}), доминирующие множества (\texttt{dominating\_number()})
    \end{itemize}
    
    \item \textbf{Статистический анализ (hypothesis\_testing.py)}:
    \begin{itemize}
        \item Критическая область: \texttt{calculate\_critical\_region()} на квантилях
        \item Мощность теста: \texttt{estimate\_power()} через сравнение с критическим значением
    \end{itemize}
    
    \item \textbf{Монте-Карло симулятор (monte\_carlo.py)}:
    \begin{enumerate}
        \item Итеративная генерация \(n\_samples\) выборок для \(H_0\) или \(H_1\)
        \item Динамическое построение графов (KNN/дистанционные)
        \item Гибкий выбор метрик через рефлексию (\texttt{getattr()})
        \item Поддержка аргументов метрик через \texttt{metric\_args}
    \end{enumerate}
\end{itemize}

\section{Описание экспериментов}

\subsection{Эксперимент 1: Зависимость характеристик от параметра $\nu$}
\label{exp1}

\textbf{Цель}: Исследовать, как характеристики графов (число треугольников для KNN, кликовое число для дистанционного) реагируют на изменение параметра $\nu$ в распределениях $\chi^2$ и $\chi$.

\begin{figure}[H]
    \centering
    \includegraphics[width=0.95\textwidth]{exp1_results.png}
    \caption{Зависимость характеристик от $\nu$ (слева — KNN-граф, справа — дистанционный)}
    \label{fig:exp1}
\end{figure}

\textbf{Ключевые наблюдения}:
\begin{itemize}
    \item \textbf{KNN-граф (число треугольников)}:
        \begin{itemize}
            \item Минимальная чувствительность: различия между $\chi^2$ и $\chi$ не превышают 0.4\% для всех $\nu$
            \item Стабильность: значения остаются в диапазоне 3012-3035 при любом $\nu$
        \end{itemize}
        
    \item \textbf{Дистанционный граф (кликовое число)}:
        \begin{itemize}
            \item Катастрофическое различие: при $\nu=3$ значения для $\chi$ в 2.13 раза выше ($113.2$ vs $53.5$)
            \item Парадоксальный рост: разрыв увеличивается с ростом $\nu$ (см. Табл.~\ref{tab:exp1})
            \item При $\nu=20$: $\chi$ показывает более чем в 5 раз большее кликовое число ($110$ vs $20$)
        \end{itemize}
\end{itemize}

\textbf{Статистика}:
\begin{table}[H]
    \centering
    \begin{tabular}{|c|c|c|c|c|}
        \hline
        $\nu$ & $H_0^{\text{DIST}}$ & $H_1^{\text{DIST}}$ & $\Delta_{\text{DIST}}$ (\%) & Отношение \\ \hline
        3 & 53.5 & 113.3 & +111.8\% & 2.12x \\ 
        5 & 38.1 & 111.2 & +191.9\% & 2.92x \\ 
        7 & 31.9 & 110.1 & +245.1\% & 3.45x \\ 
        10 & 26.9 & 110.3 & +309.7\% & 4.10x \\ 
        12 & 24.8 & 109.6 & +342.1\% & 4.42x \\ 
        15 & 22.7 & 109.4 & +381.9\% & 4.82x \\ 
        20 & 20.3 & 110.2 & +442.9\% & 5.43x \\ \hline
    \end{tabular}
    \caption{Результаты для дистанционного графа ($\Delta = \frac{|H_1 - H_0|}{H_0} \times 100\%$)}
    \label{tab:exp1}
\end{table}

\textbf{Выводы}:
\begin{itemize}
    \item \textbf{KNN-граф}:
        \begin{itemize}
            \item Полностью неэффективен для различения распределений
            \item Число треугольников практически идентично для $\chi^2$ и $\chi$
        \end{itemize}
    
    \item \textbf{Дистанционный граф}:
        \begin{itemize}
            \item Чрезвычайно чувствителен к типу распределения
            \item Эффективность растет с увеличением $\nu$
        \end{itemize}
\end{itemize}

\subsection{Эксперимент 2: Зависимость характеристик от параметров графа и размера выборки}
\label{exp2}

\textbf{Цель}: Исследовать влияние параметров графа (\(k\) для KNN, \(d\) для дистанционного) и размера выборки (\(n\)) на характеристики при фиксированных распределениях \(\chi^2(\nu=5)\) и \(\chi(\nu=5)\).

\subsubsection*{Результаты}
\begin{itemize}
    \item \textbf{KNN-граф (число треугольников)}:
        \begin{itemize}
            \item \textit{Зависимость от \(k\)}:
                \begin{itemize}
                    \item Для \(H_0\): Рост от 1,038 (\(k=5\)) до 18,526 (\(k=20\))
                    \item Для \(H_1\): Рост от 1,040 (\(k=5\)) до 18,606 (\(k=20\))
                    \item Макс. разрыв: 80.7 треугольников (\(k=20\), 0.43\%)
                \end{itemize}
            \item \textit{Зависимость от \(n\)}:
                \begin{itemize}
                    \item Для \(H_0\): Рост от 1,595 (\(n=100\)) до 7,242 (\(n=500\))
                    \item Для \(H_1\): Рост от 1,591 (\(n=100\)) до 7,259 (\(n=500\))
                    \item Разрыв < 0.23\% для всех \(n\)
                \end{itemize}
        \end{itemize}

    \item \textbf{Дистанционный граф (кликовое число)}:
        \begin{itemize}
            \item \textit{Зависимость от \(d\)}:
                \begin{itemize}
                    \item Для \(H_0\): Рост от 31.5 (\(d=0.5\)) до 97.7 (\(d=2.0\))
                    \item Для \(H_1\): Рост от 92.7 (\(d=0.5\)) до 260.4 (\(d=2.0\))
                    \item Отношение \(H_1/H_0\): от 2.94x (\(d=0.5\)) до 2.66x (\(d=2.0\))
                \end{itemize}
            \item \textit{Зависимость от \(n\)}:
                \begin{itemize}
                    \item Для \(H_0\): Рост от 57.2 (\(n=100\)) до 272.7 (\(n=500\))
                    \item Для \(H_1\): Рост от 20.7 (\(n=100\)) до 87.4 (\(n=500\))
                    \item Отношение \(H_0/H_1\): от 2.76x (\(n=100\)) до 3.12x (\(n=500\))
                \end{itemize}
        \end{itemize}
\end{itemize}

\subsubsection*{Ключевые выводы}
\begin{table}[H]
    \centering
    \begin{tabular}{|l|c|c|c|}
        \hline
        \textbf{Параметр} & \textbf{KNN (\(\Delta_{max}\), \%)} & \textbf{DIST (\(\Delta_{max}\), \%)} & \textbf{DIST (Отношение)} \\ \hline
        \(k=5 \to 20\) & 0.43 & — & — \\ 
        \(d=0.5 \to 2.0\) & — & 726.0\% & 2.94x \(\to\) 2.66x \\ 
        \(n=100 \to 500\) & 0.23 & 377.1\% & 2.76x \(\to\) 3.12x \\ \hline
    \end{tabular}
    \caption{Сводка результатов (\(\Delta = \frac{|H_1 - H_0|}{H_0} \times 100\%\))}
    \label{tab:exp2_summary}
\end{table}

\begin{itemize}
    \item \textbf{KNN-граф}:
        \begin{itemize}
            \item Число треугольников растёт с \(k\) и \(n\), но не различает \(H_0/H_1\)
            \item Максимальная разница: 0.43\% при \(k=20\)
        \end{itemize}

    \item \textbf{Дистанционный граф}:
        \begin{itemize}
            \item Кликовое число демонстрирует:
                \begin{itemize}
                    \item Максимальную чувствительность при \(d=0.5\) (\(\Delta=194.4\%\))
                    \item Стабильный рост различий с увеличением \(n\) (\(\Delta=377.1\%\))
                \end{itemize}
            \item Отношение \(H_0/H_1\) сохраняется в диапазоне 2.66x–3.12x
        \end{itemize}
\end{itemize}

\begin{table}[H]
    \centering
    \begin{tabular}{|c|c|c|c|c|}
        \hline
        \(d\) & \(H_0^{\text{DIST}}\) & \(H_1^{\text{DIST}}\) & \(\Delta_{\text{DIST}}\) (\%) & Отношение \\ \hline
        0.5 & 31.5 & 92.7 & +194.4\% & 2.94x \\ 
        1.0 & 55.0 & 164.6 & +199.3\% & 2.99x \\ 
        1.5 & 76.2 & 222.2 & +191.6\% & 2.92x \\ 
        2.0 & 97.7 & 260.4 & +166.5\% & 2.66x \\ \hline
    \end{tabular}
    \caption{Зависимость от \(d\) для дистанционного графа (\(n=300\))}
    \label{tab:dist_d}
\end{table}

\subsection{Эксперимент 3: Проверка гипотез с критической областью}
\label{exp3}

\textbf{Цель}: Оценить эффективность критериев для различения \(\chi^2(\nu=5)\) и \(\chi(\nu=5)\) при \(\alpha=0.05\).

\begin{table}[H]
    \centering
    \begin{tabular}{|l|c|c|}
        \hline
        \textbf{Метрика} & \textbf{KNN-граф} & \textbf{Дистанционный граф} \\ \hline
        Критическое значение & 7,507.15 & 97.05 \\ 
        FPR (Ошибка I рода) & 5.00\% & 5.00\% \\ 
        TPR (Мощность) & 4.80\% & 100.00\% \\ 
        AUC-ROC & 0.545 & 1.000 \\ \hline
    \end{tabular}
    \caption{Сравнение критериев (\(n=500\), \(k=10\), \(d=1.0\))}
    \label{tab:exp3_results}
\end{table}

\subsubsection*{Анализ результатов}
\begin{itemize}
    \item \textbf{KNN-граф (число треугольников)}:
        \begin{itemize}
            \item Низкая мощность (4.8\%): Менее 5\% выборок \(H_1\) попадают в критическую область
            \item AUC 0.545: Незначительное улучшение над случайным угадыванием (0.5)
            \item FPR строго соответствует уровню \(\alpha=0.05\)
        \end{itemize}
        
    \item \textbf{Дистанционный граф (кликовое число)}:
        \begin{itemize}
            \item Идеальная сепарация: AUC=1.0 и мощность=100\% 
            \item Все выборки \(H_1\) превышают критическое значение
            \item Стабильный контроль ошибки I рода (ровно 5\%)
        \end{itemize}
\end{itemize}

\subsubsection*{Практические выводы}
\begin{itemize}
    \item Дистанционный граф с характеристикой "кликовое число" демонстрирует:
        \begin{itemize}
            \item Абсолютную надежность при \(d=1.0\)
            \item Эффективный контроль ошибок обоих типов
        \end{itemize}
    \item KNN-граф требует:
        \begin{itemize}
            \item Пересмотра используемой характеристики (число треугольников неинформативно)
            \item Дополнительных исследований для поиска значимых метрик
        \end{itemize}
    \item Оптимальная конфигурация: \(d=1.0\), \(n \geq 500\) гарантирует AUC=1.0
\end{itemize}

\section*{Заключение (Часть I)}
\begin{itemize}
    \item KNN-граф не подходит для проверки гипотез в текущей конфигурации.
    \item Дистанционный граф с характеристикой «кликовое число» показал идеальное разделение (\(AUC = 1.0\)).
    \item Возможно, для KNN-графа стоит изучить другие характеристики.
\end{itemize}

\newpage

\section*{Часть II: Анализ графовых признаков для классификации распределений}

\section{Введение}
Цель исследования --- оценить эффективность графовых признаков, построенных на выборках из распределений $\chi^2(5)$ и $\chi(5)$, для задачи бинарной классификации.

\section{Описание экспериментов}
\subsection{Извлечение признаков}
Для каждой выборки размера $n$ строился дистанционный граф с порогом $d=1.0$ и вычислялись четыре признака.

\subsection{Анализ важности признаков}
При помощи RandomForest оценивалась важность признаков при $n=25,100,500$. Результаты приведены в таблице:
\begin{table}[H]
\centering
\begin{tabular}{lrrr}
\toprule
Признак & $n=25$ & $n=100$ & $n=500$ \\
\midrule
count\_triangles       & 0.49 & 0.45 & 0.45 \\
clique\_number         & 0.34 & 0.39 & 0.39 \\
min\_degree            & 0.00 & 0.01 & 0.05 \\
connected\_components  & 0.16 & 0.15 & 0.11 \\
\bottomrule
\end{tabular}
\caption{Важность признаков при разных размерах выборки}
\label{tab:feature_importance}
\end{table}
Вывод: \texttt{count\_triangles} и \texttt{clique\_number} являются наиболее информативными.

\subsection{Классификация и метрики качества}
Эксперименты проводились для $n=10,20,50,100,200,500$ с классификаторами 
LogisticRegression, RandomForest и SVM. Оценивались Accuracy, дисперсия Accuracy, FPR, TPR, Precision и F1.

\begin{figure}[H]
\centering
\includegraphics[width=0.9\textwidth]{classification_metrics.png}
\caption{Зависимость метрик качества от размера выборки}
\label{fig:metrics}
\end{figure}

\section{Выводы (Часть II)}
\begin{itemize}
\item При $n\ge20$ все алгоритмы достигают 100\% Accuracy и мощности, при этом FPR = 0.
\item Для практических задач достаточно $n\approx20\text{-}50$ для идеального разделения.
\item RandomForest и SVM показали наилучшую стабильность при малых выборках.
\item Наиболее информативные признаки: \texttt{count\_triangles} и \texttt{clique\_number}.
\end{itemize}

\end{document}