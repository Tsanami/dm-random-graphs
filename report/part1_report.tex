\documentclass[12pt,a4paper]{article}
\usepackage[utf8]{inputenc}
\usepackage[T2A]{fontenc}
\usepackage[russian]{babel}
\usepackage{geometry}
\usepackage{amsmath,amssymb}
\usepackage{booktabs}
\usepackage{graphicx}
\usepackage{hyperref}
\geometry{margin=2.5cm}

\title{Отчёт по проверке гипотез \\ c использованием случайных графов \\ \small Часть I: Stable$(\alpha=1)$ vs Normal$(0,1)$}
\author{Хамаганов Ильдар \\}
\date{}

\begin{document}
\maketitle

\section*{Введение}
Целью данной части исследования было оценить, насколько топологические характеристики случайных графов позволяют различать выборки из двух распределений:
\begin{itemize}
  \item H$_0$: Stable$(\alpha=1)$;
  \item H$_1$: Normal$(0,1)$.
\end{itemize}
Использовались два типа графов:
\begin{description}
  \item[KNN‑граф:] характеристика $T^{\rm knn} = \max\deg(G)$ (максимальная степень).
  \item[Дистанционный граф:] характеристика $T^{\rm dist} = \chi(G)$ (хроматическое число).
\end{description}

\section{Настройка окружения и код}
\subsection*{Импорт и автозагрузка}
\begin{verbatim}
%load_ext autoreload
%autoreload 2
import sys, os
project_root = os.path.abspath(os.path.join(os.getcwd(),'..'))
sys.path.append(project_root)

import numpy as np
import matplotlib.pyplot as plt
import seaborn as sns
from tqdm import tqdm

from src.data_utils import sample_stable, sample_normal
from src.build_graph import build_knn_graph, build_distance_graph
from src.graph_analyzer import GraphAnalyzer
from src.monte_carlo import monte_carlo_simulation
from src.visualization import plot_distributions, plot_critical_region
\end{verbatim}

\subsection*{Параметры экспериментов}
\begin{itemize}
  \item Размер выборки: $n=200$.
  \item Число MC‑итераций: $N_{\rm MC}=500$.
  \item Параметры KNN‑графа: $k\in\{3,5,7,10,12,15,20\}$.
  \item Параметры дистанционного графа: $d\in\{0.5,1.0,1.5,2.0\}$.
\end{itemize}

\section{Эксперимент1: зависимость от «ν»}
\paragraph{Описание.}
Для каждой из двух распределений (Stable, Normal) вычисляли:
\[
\overline{T}^{\rm knn}(k)
\;=\;\text{mean}\bigl[\max\deg(G)\bigr],\quad
\overline{T}^{\rm dist}(d)
\;=\;\text{mean}\bigl[\chi(G)\bigr].
\]

\paragraph{Результаты.}
\begin{figure}[ht]
  \centering
  \includegraphics[width=0.48\textwidth]{delta_vs_k.png}
  \includegraphics[width=0.48\textwidth]{chi_vs_d.png}
  \caption{Слева: $\overline{T}^{\rm knn}=\Delta(G)$ vs $k$. 
           Справа: $\overline{T}^{\rm dist}=\chi(G)$ vs $d$.}
\end{figure}

\paragraph{Выводы.}
\begin{itemize}
  \item \textbf{KNN‑граф ($T^{\rm knn}$)}: кривая почти горизонтальна, 
    разрыв между Stable и Normal менее 1\%, распределения перекрываются.
  \item \textbf{Дистанционный граф ($T^{\rm dist}$)}: $\chi(G)$ растёт с $d$, 
    и для Normal значения значительно выше (до $\sim$140 vs $\sim$103 при $d=2$). 
    Статистика хорошо разделяет гипотезы.
\end{itemize}

\section{Эксперимент 2: зависимость от $k,d,n$}
\paragraph{Описание.}
Исследовали:
\begin{enumerate}
  \item Зависимость $\overline{T}^{\rm knn}(k)$ и $\overline{T}^{\rm dist}(d)$ при $n=200$.
  \item Зависимость при фиксированных $k=10$, $d=1.0$ от $n\in\{100,200,300,500\}$.
\end{enumerate}

\paragraph{Сводные итоги.}
\begin{table}[ht]
\centering
\caption{Отношение $\overline{T}^{H_1}/\overline{T}^{H_0}$}
\begin{tabular}{lccc}
\toprule
Параметр & KNN ($k$) & Dist ($d$) & Dist ($n$) \\
\midrule
Минимум & 0.92× & 1.47× & 2.30× \\
Максимум & 1.06× & 2.80× & 3.10× \\
\bottomrule
\end{tabular}
\end{table}

\paragraph{Выводы.}
\begin{itemize}
  \item $\Delta(G)$ увеличивается с $k,n$, но H$_0$/H$_1$ остаются близки (отношения $\approx 0.9$–1.06).
  \item $\chi(G)$ показывает высокую чувствительность: отношение до $\sim3\times$ при росте $n$.
\end{itemize}

\section{Эксперимент 3: критические области и мощность}
\paragraph{Условия.} $n=500$, $k=10$, $d=1.0$, уровень значимости $\alpha=0.05$.

\paragraph{Результаты.}
\begin{table}[ht]
\centering
\caption{Критические значения и характеристики теста}
\begin{tabular}{lcccc}
\toprule
Граф & CV & FPR & TPR & AUC \\
\midrule
KNN ($\Delta$)    & 18.7  & 5.0\%  & 4.8\%   & 0.545 \\
Distance ($\chi$) & 114   & 5.0\%  & 100.0\% & 1.000 \\
\bottomrule
\end{tabular}
\end{table}

\paragraph{Выводы.}
\begin{itemize}
  \item Тест по $\Delta(G)$ практически не различает гипотезы (мощность $\approx\alpha$).
  \item Тест по $\chi(G)$ идеально разделяет выборки (AUC=1, мощность=100\%).
\end{itemize}

\section*{Заключение}
\begin{itemize}
  \item Для Stable$(1)$ vs Normal$(0,1)$ лучшая статистика — \(\chi(G)\) дистанционного графа.
  \item $\Delta(G)$ KNN‑графа неинформативна и не подходит в качестве единственной характеристики.
  \item Рекомендуем: $d=1.0$, $n\ge500$ для надёжного критерия.
  \item Возможные улучшения: исследовать дополнительные метрики (центральность, доминирование).
\end{itemize}

\end{document}
